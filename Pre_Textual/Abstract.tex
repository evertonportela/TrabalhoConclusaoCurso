% resumo em inglês
\setlength{\absparsep}{18pt} % ajusta o espaçamento dos parágrafos do resumo
\begin{resumo}[Abstract]
 \begin{otherlanguage*}{english}
   
   According to the Ministry of Health (2016), individuals in hospital stay do not eat adequately, causing risks linked to malnutrition, which influence clinical complications, longer hospitalization, readmission cases and higher hospital costs. The intense investments in the use of digital technologies in the area of health have been an important tool to help health professionals to improve human health and well-being, health care services and prevent medical errors. This work aims to present a Business Intelligence (BI) environment, focused on hospital management in the Clinical Nutrition Unit of the University Hospital of Aracaju. The use of BI allows an analytical analysis and monitoring of nutritional care information generated from data from the department itself to support decision-making, assisting hospital management in better conducting the provision of hospital services.

   \vspace{\onelineskip}
 
   \noindent 
   \textbf{Keywords}: Decision Support System. Support to Decision Making. Data analysis. Data Warehouse. OLAP. Nutritional screening. and health. Nutritional Monitoring. Hospital. KPI.
 \end{otherlanguage*}
\end{resumo}