% resumo em inglês
\setlength{\absparsep}{18pt} % ajusta o espaçamento dos parágrafos do resumo
\begin{resumo}[Abstract]
 \begin{otherlanguage*}{english}
   
Studies indicate that hospitalized individuals have a high risk of developing malnutrition, which influences other clinical complications and consequently results in longer hospitalization and higher hospital costs. Intense investments in Information and Communication Technologies have been an important differential, considering the advances in the quality of health services and greater accuracy in the diagnosis of diseases, also combining the tools of manipulation and analysis of data to help health professionals improve human health and well-being, health care services and the prevention of medical errors. This paper presents a systematic review carried out to know the management and analysis characteristics of the decision support systems for existing hospital nutrition and a Business Intelligence (BI) solution, focused on the management of the clinical nutrition unit of the University Hospital of Aracaju. The solution used in its structure a Data Warehouse to store information from different data sources, a Data Mart, and a logical data cube for dimensional queries. Tied to this, pentaho's platform was used for data extraction, processing and loading and graphical display of information. The use of BI allowed the nutrition unit to analyze and monitor nutritional care information more quickly and with a greater analytical field of vision. 

   \vspace{\onelineskip}
 
   \noindent 
   \textbf{Keywords}: Decision Support Systems. Data Warehouse. Business Intelligence. Nutritional Monitoring.
 \end{otherlanguage*}
\end{resumo}