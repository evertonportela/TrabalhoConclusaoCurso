% resumo em inglês
\setlength{\absparsep}{18pt} % ajusta o espaçamento dos parágrafos do resumo
\begin{resumo}[Abstract]
 \begin{otherlanguage*}{english}
   
Studies indicate that hospitalized individuals have a high risk of developing malnutrition, which influences other clinical complications and consequently results in longer hospitalization and higher hospital costs. The intense investments in the use of digital technologies in the health area have been an important tool to help professionals to improve human health and well-being, health care services and prevent medical errors. This paper presents a solution of textit{Business Intelligence} (BI), focused on hospital management in the clinical nutrition unit of the University Hospital of Aracaju. The solution used in its framework a textit{Data Warehouse} to store information from different data sources, a textit{Data Mart} and a logical data cube for dimensional queries. Tied to this, pentaho's platform was used for data extraction, processing and loading and graphical display of information. The use of BI allowed the department to analyze and monitor nutritional care information, which was generated from data from the unit itself, faster and with a greater analytical field of vision.

   \vspace{\onelineskip}
 
   \noindent 
   \textbf{Keywords}: Decision Support System. Data Warehouse. Olap. e-Health. Nutritional Monitoring..
 \end{otherlanguage*}
\end{resumo}