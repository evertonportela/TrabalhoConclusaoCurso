% resumo em português
\setlength{\absparsep}{18pt} % ajusta o espaçamento dos parágrafos do resumo
\begin{resumo}

Estudos apontam que indivíduos hospitalizados têm grande risco de desenvolver desnutrição, que influencia outras complicações clínicas e consequentemente resultam num maior tempo de internação e maiores custos hospitalares. 
Intensos investimentos em Tecnologias da Informação e Comunicação têm sido importante diferencial, considerando os avanços na qualidade dos serviços de saúde e maior precisão no diagnóstico de doenças, aliando ainda as ferramentas de manipulação e análise de dados visando ajudar profissionais da saúde a melhorarem a saúde e o bem-estar humano, os serviços de atenção à saúde e prevenirem erros médicos. 
Este trabalho apresenta uma revisão sistemática realizada para conhecer as características de gerenciamento e análise dos sistemas de apoio a decisão para nutrição hospitalar existentes e uma solução de \textit{Business Intelligence} (BI), voltada para a gestão da unidade de nutrição clínica do Hospital Universitário de Aracaju. A solução utilizou em sua estrutura um \textit{Data Warehouse} para armazenamento das informações advindas de diferentes fontes de dados, um \textit{Data Mart} e um cubo lógico de dados para as consultas dimensionais. 
Atrelado a isto, foi usada a plataforma da Pentaho para extração, tratamento e carga dos dados e exibição gráfica das informações. A utilização do BI permitiu a unidade de nutrição analisar e monitorar as informações de cuidado nutricional de modo mais rápido e com maior campo de visão analítico. 


 \textbf{Palavras-chave}: Sistemas de Apoio à Decisão. \textit{Data Warehouse}. \textit{Business Intelligence}. Acompanhamento Nutricional.
\end{resumo}