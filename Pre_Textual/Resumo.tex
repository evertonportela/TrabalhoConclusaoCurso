% resumo em português
\setlength{\absparsep}{18pt} % ajusta o espaçamento dos parágrafos do resumo
\begin{resumo}
 
Segundo o \citeonline{manualnutricao2016}, indivíduos em período de internação hospitalar não se alimentam adequadamente, causando riscos atrelados a desnutrição, que influenciam complicações clínicas, maior tempo de internação, casos de reinternação e maiores custos hospitalares. Os intensos investimentos no uso das tecnologias digitais na área da saúde têm sido uma importante ferramenta de ajuda a profissionais da saúde para melhorarem a saúde e o bem-estar humano, os serviços de atenção a saúde e prevenirem erros médicos. Este trabalho tem por objetivo apresentar um ambiente de \textit{Business Intelligence} (BI), voltado para a gestão hospitalar na Unidade de Nutrição Clínica do Hospital Universitário de Aracaju. A utilização do BI permite uma análise analítica e monitoramento das informações de cuidado nutricional geradas a partir de dados do próprio departamento para apoio à tomada de decisões, auxiliando a gestão hospitalar numa melhor condução da prestação dos serviços hospitalares. 

 \textbf{Palavras-chave}: Sistema de Apoio a Decisão. Apoio a Tomada de Decisão. Análise de Dados. Data Warehouse. OLAP. Triagem Nutricional. e-Saúde. Acompanhamento Nutricional. Hospital. KPI.
\end{resumo}