% resumo em português
\setlength{\absparsep}{18pt} % ajusta o espaçamento dos parágrafos do resumo
\begin{resumo}
 
Estudos apontam que indivíduos hospitalizados têm grande risco de desenvolver desnutrição, que influencia outras complicações clínicas e consequentemente resultam num maior tempo de internação e maiores custos hospitalares. Os intensos investimentos no uso das tecnologias digitais na área da saúde têm sido uma importante ferramenta de ajuda a profissionais para melhorarem a saúde e o bem-estar humano, os serviços de atenção à saúde e prevenirem erros médicos. Este trabalho apresenta uma solução de \textit{Business Intelligence} (BI), voltada para a gestão hospitalar na unidade de nutrição clínica do Hospital Universitário de Aracaju. A solução utilizou em sua estrutura um \textit{Data Warehouse} para armazenamento das informações advindas de diferentes fontes de dados, um \textit{Data Mart} e um cubo lógico de dados para as consultas dimensionais. Atrelado a isto, foi usada a plataforma da Pentaho para extração, tratamento e carga dos dados e exibição gráfica das informações. A utilização do BI permitiu ao departamento analisar e monitorar as informações de cuidado nutricional, que foram geradas a partir de dados da própria unidade, de modo mais rápido e com maior campo de visão analítico. 

 \textbf{Palavras-chave}: Sistemas de Apoio à Decisão. \textit{Data Warehouse}. OLAP. e-Saúde. Acompanhamento Nutricional.
\end{resumo}