\chapter{Introdução}

Este documento e seu código-fonte são exemplos de referência de uso da classe \textbf{abntex2} e do pacote \textbf{abntex2cite}. O documento exemplifica a elaboração de trabalho acadêmico (tese, dissertação e outros do gênero) produzido conforme a ABNT NBR 14724:2011 \emph{Informação e documentação - Trabalhos acadêmicos - Apresentação}.

A expressão ``Modelo Canônico'' é utilizada para indicar que \abnTeX\ não é modelo específico de nenhuma universidade ou instituição, mas que implementa tão somente os requisitos das normas da ABNT. Uma lista completa das normas observadas pelo \abnTeX\ é apresentada em \citeonline{abntex2classe}.

Sinta-se convidado a participar do projeto \abnTeX! Acesse o site do projeto em \url{http://www.abntex.net.br/}. Também fique livre para conhecer, estudar, alterar e redistribuir o trabalho do \abnTeX, desde que os arquivos modificados tenham seus nomes alterados e que os créditos sejam dados aos autores originais, nos termos da ``The \LaTeX\ Project Public License''\footnote{\url{http://www.latex-project.org/lppl.txt}}.

Encorajamos que sejam realizadas customizações específicas deste exemplo para universidades e outras instituições --- como capas, folha de aprovação, etc. Porém, recomendamos que ao invés de se alterar diretamente os arquivos do \abnTeX, distribua-se arquivos com as respectivas customizações. Isso permite que futuras versões do \abnTeX~não se tornem automaticamente incompatíveis com as customizações promovidas. Consulte \citeonline{abntex2-wiki-como-customizar} para mais informações.

Este documento deve ser utilizado como complemento dos manuais do \abnTeX\ \cite{abntex2classe,abntex2cite,abntex2cite-alf} e da classe \textsf{memoir} \cite{memoir}. 

Esperamos, sinceramente, que o \abnTeX\ aprimore a qualidade do trabalho que você produzirá, de modo que o principal esforço seja concentrado no principal: na contribuição científica. 

Equipe \abnTeX. Lauro César Araújo

% ---
\section{Objetivos}\label{sec-divisoes}
% ---
Nesta seção são descritos os objetivos gerais e específicos para realização do corrente trabalho de conclusão de curso.

\subsection{Geral}\label{sec-divisoes-subsection}
Desenvolver um sistema de apoio à decisão que auxilie no processo de tomada de decisão de gestores da área nutricional, fornecendo informação para que respondam rapidamente as necessidades inerentes às suas atividades no ambiente hospitalar.
\subsection{Específicos}\label{sec-divisoes-subsection}
Para alcançar o objetivo acima citado, foram propostos os seguintes objetivos específicos:
\begin{itemize}
 \item Realizar um mapeamento sistemático para obter o estado da arte de publicações sobre a aplicação do conceito de sistemas  de apoio à decisão no ambiente nutricional hospitalar;

 \item Mapear os indicadores e informações necessárias para o \textit{dashboard};

 \item Desenvolver um sistema de apoio a decisão;

 \item Testar e validar o software com o objetivo de obter o \textit{feedback} dos possíveis usuários, utilizando o Hospital Universitário da Universidade Federal de Sergipe como estudo de caso.
\end{itemize}

% ---
\section{Metodologia}\label{sec-divisoes}
% ---
Com base nos objetivos definidos esta pesquisa pode ser classificada como pesquisa exploratória, um tipo de pesquisa que visa proporcionar, segundo \citeonline[p.~41]{gil2002}, maior familiaridade com o problema, com vistas a torná-lo mais explícito ou constituir hipóteses.

Também se caracteriza como pesquisa bibliográfica, em razão ao levantamento bibliográfico realizado no Mapeamento Sistemático proposto, tendo como base material já elaborado, constituído principalmente de livros e artigos científicos \cite{gil2002}.

O sistema foi desenvolvido utilizando informações e termos técnicos em conformidade com as recomendações presentes no Manual de Terapia Nutricional do SUS \cite{manualnutricao2016}

\textbf{INSERIR CONFORMIDADES DE DESENVOLVIMENTO - MANUAL DE TERAPIA NUTRICIONAL  (MINISTERIO DA SAUDE 2006)}
\textbf{INSERIR DESENVOLVIMENTO DA APLICACAO (PRESSMAN 2011; SOMMERVILLE 2011)}

% ---
\section{Estrutura do Documento}\label{sec-divisoes}
% ---
Para melhor navegação e entendimento, este documento está estruturado em \textbf{X} capítulos além desta introdução. Os tópicos a seguir descrevem o conteúdo de cada capítulo, sendo eles:
\begin{itemize}
 \item Capítulo 2 - Fundamentação Teórica;

 \item Capítulo 3 - Mapeamento Sistemático;

 \item Capítulo 4 - Desenvolvimento;
 
 \item Capítulo 5 - Considerações Finais.
\end{itemize}