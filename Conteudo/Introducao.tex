\chapter{Introdução}
Segundo o \citeonline{manualnutricao2016}, pesquisas indicam que alguns indivíduos não se alimentam adequadamente durante o período de internação hospitalar, o que acarreta um estado de desnutrição, que influencia no aumento do risco de complicações pós-operatórias, doenças relacionadas como lesões e infecções e é associada ao aumento do tempo de internação, reinternação e maiores custos hospitalares.

São necessários critérios que identifiquem o risco nutricional na admissão do paciente e durante sua permanência no hospital, sendo assim, necessário implementá-los na rotina hospitalar, visto que a perda de nutrientes essenciais ao organismo, pode ocorrer durante o período de hospitalização. Identificar a desnutrição é um objetivo essencial a ser alcançado no tratamento geral de pacientes hospitalizados. O diagnóstico adequado é fundamental para que a terapia nutricional possa ser realizada com qualidade e iniciada o mais rápido possível \cite{keller2014}.

Considerando avanços na qualidade dos serviços de saúde e maior precisão no diagnóstico e no tratamento de doenças, o setor de saúde tem seguindo uma tendência mundial e intensificado investimentos em Tecnologia de Informação e Comunicação (TIC) nos estabelecimentos de saúde \cite{saudedigital2019}. O uso dessas tecnologias é ainda, segundo a \citeonline{ehealth}, definida como e-Saúde e inclui tanto a saúde móvel (\textit{mHealth} em inglês) como a saúde eletrônica (\textit{eHealth} em inglês). A saúde digital contempla o uso geral das TIC, para melhorar a saúde humana, os serviços de atenção à saúde e o bem-estar de indivíduos e populações, sendo uma importante ferramenta para ajudar os profissionais da saúde a oferecerem os melhores serviços e prevenirem erros médicos.

Este trabalho descreve o desenvolvimento de uma solução de \textit{Business Intelligence} voltado para a Unidade de Nutrição Clínica do Hospital Universitário de Aracaju. Cujo objetivo é apoiar o processo de tomada de decisão da gerência da unidade e do hospital, oferecendo informações sobre os índices alcançados pela unidade no cuidado nutricional.  


% ---
\section{Objetivos}\label{sec-divisoes}
% ---
Nesta seção são descritos os objetivos gerais e específicos para realização do corrente trabalho de conclusão de curso.

\subsection{Geral}\label{sec-divisoes-subsection}
Desenvolver uma solução de \textit{Business Intelligence} que auxilie no processo de tomada de decisão de gestores da área nutricional, fornecendo informação para que respondam rapidamente as necessidades inerentes às suas atividades no ambiente hospitalar.

\subsection{Específicos}\label{sec-divisoes-subsection}
Para alcançar o objetivo acima citado, foram propostos os seguintes objetivos específicos:
\begin{itemize}
 \item Realizar uma revisão sistemática para obter o estado da arte de publicações sobre a aplicação do conceito de sistemas de apoio à decisão no ambiente nutricional hospitalar;

 \item Mapear os indicadores e informações necessárias para os \textit{dashboards};

 \item Elaborar um sistema \textit{Extract - Transform - Load} (ETL) com os dados disponibilizados pela Unidade de Nutrição Clínica;
 
 \item Elaborar um modelo multidimensional de dados;
 
 \item Disponibilizar uma ferramenta de consulta OLAP para auxiliar a tomada de decisão da gestão da Unidade de Nutrição e do Hospital Universitário.
\end{itemize}

% ---
\section{Metodologia}\label{sec-divisoes}
% ---
Com base nos objetivos definidos esta pesquisa pode ser classificada como pesquisa exploratória, um tipo de pesquisa que visa proporcionar, segundo \citeonline{gil2002}, maior familiaridade com o problema, com vistas a torná-lo mais explícito ou constituir hipóteses.
Também se caracteriza como pesquisa bibliográfica, em razão ao levantamento bibliográfico realizado na Revisão Sistemática proposta, tendo como base material já elaborado, constituído principalmente de livros e artigos científicos \cite{gil2002}.

A solução de \textit{Business Intelligence} foi implementada utilizando informações e termos técnicos em conformidade com as recomendações presentes no Manual de Terapia Nutricional do SUS \cite{manualnutricao2016}.

% ---
\section{Estrutura do Documento}\label{sec-divisoes}
% ---
Para melhor navegação e entendimento, este documento está estruturado em 5 capítulos, além desta introdução. Os tópicos a seguir descrevem o conteúdo de cada capítulo:
\begin{itemize}
 \item Capítulo 2 - Fundamentação Teórica: relata sobre conceitos fundamentais necessários para o entendimento do trabalho, como terapia nutricional, \textit{Business Intelligence} e modelagem multidimensional;

 \item Capítulo 3 - Revisão Sistemática: apresenta o processo metodológico, os resultados e a análise dos artigos selecionados indicando o estado da arte sobre o uso de apoio a decisão para nutrição clínica hospitalar;

 \item Capítulo 4 - Desenvolvimento: discorre sobre os detalhes do \textit{Data Warehouse} e do \textit{Data Mart} construídos, o cubo lógico de dados, o processo de ETL, construção do BI e as consultas implementadas;
 
 \item Capítulo 5 - Considerações Finais: descreve uma visão geral do projeto desenvolvido resumindo todas as informações relevantes. Possíveis melhorias para trabalhos futuros são descritas ao final do capítulo.
\end{itemize}